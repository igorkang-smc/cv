%-------------------------
% Resume in Latex
% Author : igor kan
% Based off of: https://github.com/sb2nov/resume
% License : MIT
%------------------------

\documentclass[letterpaper,11pt]{article}

\usepackage{latexsym}
\usepackage[empty]{fullpage}
\usepackage{titlesec}
\usepackage{marvosym}
\usepackage[usenames,dvipsnames]{color}
\usepackage{verbatim}
\usepackage{enumitem}
\usepackage[hidelinks]{hyperref}
\usepackage{fancyhdr}
\usepackage[english]{babel}
\usepackage{tabularx}
\usepackage{kotex}
\usepackage[normalem]{ulem}   % in preamble

%----------FONT OPTIONS----------
% sans-serif
% \usepackage[sfdefault]{FiraSans}
% \usepackage[sfdefault]{roboto}
% \usepackage[sfdefault]{noto-sans}
% \usepackage[default]{sourcesanspro}

% serif
% \usepackage{CormorantGaramond}
% \usepackage{charter}


\pagestyle{fancy}
\fancyhf{} % clear all header and footer fields
\fancyfoot{}
\renewcommand{\headrulewidth}{0pt}
\renewcommand{\footrulewidth}{0pt}

% Adjust margins
\addtolength{\oddsidemargin}{-0.5in}
\addtolength{\evensidemargin}{-0.5in}
\addtolength{\textwidth}{1in}
\addtolength{\topmargin}{-.5in}
\addtolength{\textheight}{1.0in}

\urlstyle{same}

\raggedbottom
\raggedright
\setlength{\tabcolsep}{0in}

% Sections formatting
\titleformat{\section}{
    \vspace{-4pt}\scshape\raggedright\large
}{}{0em}{}[\color{black}\titlerule \vspace{-5pt}]

% Ensure that generate pdf is machine readable/ATS parsable
\pdfgentounicode=1

%-------------------------
% Custom commands
\newcommand{\resumeItem}[1]{
    \item\small{
            {#1 \vspace{-2pt}}
    }
}

\newcommand{\resumeSubheading}[4]{
    \vspace{-2pt}\item
    \begin{tabular*}{0.97\textwidth}[t]{l@{\extracolsep{\fill}}r}
    \textbf{#1} & #2 \\
    \textit{\small#3} & \textit{\small #4} \\
    \end{tabular*}\vspace{-7pt}
}

\newcommand{\resumeSubSubheading}[2]{
    \item
    \begin{tabular*}{0.97\textwidth}{l@{\extracolsep{\fill}}r}
    \textit{\small#1} & \textit{\small #2} \\
    \end{tabular*}\vspace{-7pt}
}

\newcommand{\resumeProjectHeading}[2]{
    \item
    \begin{tabular*}{0.97\textwidth}{l@{\extracolsep{\fill}}r}
    \small#1 & #2 \\
    \end{tabular*}\vspace{-7pt}
}

\newcommand{\resumeSubItem}[1]{\resumeItem{#1}\vspace{-4pt}}

\renewcommand\labelitemii{$\vcenter{\hbox{\tiny$\bullet$}}$}

\newcommand{\resumeSubHeadingListStart}{\begin{itemize}[leftmargin=0.15in, label={}]}
\newcommand{\resumeSubHeadingListEnd}{\end{itemize}}
\newcommand{\resumeItemListStart}{\begin{itemize}}
\newcommand{\resumeItemListEnd}{\end{itemize}\vspace{-5pt}}

%-------------------------------------------
%%%%%%  RESUME STARTS HERE  %%%%%%%%%%%%%%%%%%%%%%%%%%%%


\begin{document}

%----------HEADING----------
% \begin{tabular*}{\textwidth}{l@{\extracolsep{\fill}}r}
%   \textbf{\href{http://sourabhbajaj.com/}{\Large Sourabh Bajaj}} & Email : \href{mailto:sourabh@sourabhbajaj.com}{sourabh@sourabhbajaj.com}\\
%   \href{http://sourabhbajaj.com/}{http://www.sourabhbajaj.com} & Mobile : +1-123-456-7890 \\
% \end{tabular*}

    \begin{center}
        \textbf{\Huge \scshape 강이고르} \\ \vspace{1pt}
        \small 010-3375-5331 $|$
        \href{mailto:x@x.com}{\uline{ikanv13@gmail.com}} $|$
        \href{https://linkedin.com/in/rocket13}{\uline{linkedin.com/in/rocket13}} $|$
        \href{https://github.com/igorkang-smc}{\uline{github.com/igorkang-smc}} $|$
        \href{https://igorkang.notion.site/Portfolio-25bc1bafd5148063a160eb2fd609a4f1}{\uline{portfolio}}
    \end{center}


%-----------EDUCATION-----------
    \section{학력사항}
    \resumeSubHeadingListStart
    \resumeSubheading
    {로모노소프 명칭의 모스크바 국립대학교}{타슈켄트, UZ}
    {응용수학 및 컴퓨터 과학 (학사 학위)}{2015년 9월 – 2019년 6월}
    \resumeSubHeadingListEnd


%-----------EXPERIENCE-----------
    \section{경력사항}
    \resumeSubHeadingListStart

    \resumeSubheading
    {백엔드 개발자}{2022년 4월 – 2025년 3월}
    {주식회사 더에스엠씨그룹}{서울, 한국}
    \resumeItemListStart
    \resumeItem{AI 에이전트·챗봇·내부 AI 앱 설계·배포로 R\&D 속도 향상}
    \resumeItem{AWS Lambda, FastAPI, Laravel, Django로 다양한 프로젝트의 확장형 API/백엔드 구현, 1000+ 동시 사용자 지원}
    \resumeItem{Kubernetes, AWS, Docker, Terraform 기반 인프라 CI/CD 자동화, 배포 시간 40\% 단축}
    \resumeItemListEnd

% -----------Multiple Positions Heading-----------
%    \resumeSubSubheading
%     {Software Engineer I}{Oct 2014 - Sep 2016}
%     \resumeItemListStart
%        \resumeItem{Apache Beam}
%          {Apache Beam is a unified model for defining both batch and streaming data-parallel processing pipelines}
%     \resumeItemListEnd
%    \resumeSubHeadingListEnd
%-------------------------------------------

    \resumeSubheading
    {개발팀 주임}{2021년 9월 – 2022년 2월}
    {주식회사 태인교육}{성남, 한국}
    \resumeItemListStart
    \resumeItem{Java Spring Framework로 핵심 제품을 구축하고 플랫폼 안정성 향상에 기여하였습니다}
    \resumeItem{CSS, JS(jQuery), React, Vue.js로 SPA·랜딩·백오피스 등 반응형 프론트엔드 개발}
    \resumeItem{다양한 부서와 고객과의 효과적인 협업을 통해 프로젝트 전달을 조율하였습니다}
    \resumeItemListEnd

    \resumeSubheading
    {주니어 개발자}{2020년 1월 – 2021년 1월}
    {주식회사 클릭}{타슈켄트, 우즈베키스탄}
    \resumeItemListStart
    \resumeItem{중요 업무용 결제 시스템과 핸들러를 유지 관리하며 안정적인 거래를 지원하였습니다}
    \resumeItem{결제 게이트웨이 테스트를 포함하여 기술 지원 및 시스템 통합을 수행하였습니다}
    \resumeItem{비즈니스 운영을 효율화하기 위해 백오피스 솔루션을 개발하고 관리하였습니다}
    \resumeItemListEnd

    \resumeSubHeadingListEnd


%-----------PROJECTS-----------
    \section{프로젝트}
    \resumeSubHeadingListStart
    \resumeProjectHeading
    {\textbf{스마트 공간 AI} $|$ \emph{Python, Fast API, Langchain, OpenAI API, PostgreSQL, Docker}}{2024년 7월 – 2025년 3월}
    \resumeItemListStart
    \resumeItem{FastAPI로 웹 백엔드를 개발하고, OpenAI 모델과 LangChain SQLAgent를 통합한 REST API를 제공하였습니다}
    \resumeItem{두 개의 사무실 전역에서 예약 가능한 모든 공간을 관리하기 위한 통합 데이터베이스를 구축 및 운영하였습니다}
    \resumeItem{서비스를 배포하고 모니터링 및 신속한 재배포를 위한 CI/CD 파이프라인을 구현하였습니다}
    \resumeItemListEnd
    \resumeProjectHeading
    {\textbf{던킨 홈페이지} $|$ \emph{Laravel, MySQL, AWS, Inertia.js, Git}}{2023년 8월 – 2024년 4월}
    \resumeItemListStart
    \resumeItem{Laravel과 Inertia.js 기반으로 백엔드를 직접 개발하며 외부 디자이너 및 프론트엔드 팀과 긴밀히 협업}
    \resumeItem{AWS 인프라 위에 애플리케이션을 배포하고 서버 및 데이터베이스를 포함한 환경을 설정했습니다}
    \resumeItem{새로운 기능과 UI 개선을 지속적으로 반영해 안정적인 성능을 유지하고 고객 만족을 달성}
    \resumeItemListEnd
    \resumeProjectHeading
    {\textbf{배스킨라빈스 홈페이지} $|$ \emph{AWS, PHP, JS, Git}}{2023년 1월 – 2023년 6월}
    \resumeItemListStart
    \resumeItem{국내 대표 아이스크림 브랜드의 트래픽이 많은 웹사이트를 유지 보수하고 기능을 확장하였습니다}
    \resumeItem{Happy Point 인증 연동 및 고객 피드백/불만 처리 시스템 등 주요 기능을 구현하였습니다}
    \resumeItem{AWS 인프라에서 애플리케이션을 운영하여 안정성과 신뢰성 있는 서비스를 보장하였습니다}
    \resumeItemListEnd
    \resumeSubHeadingListEnd



%
%-----------PROGRAMMING SKILLS-----------
    \section{보유기술}
    \begin{itemize}[leftmargin=0.15in, label={}]
        \small{\item{
            \textbf{언어}{: Python, PHP, Java, SQL, NoSQL, JavaScript, HTML/CSS} \\
            \textbf{프레임워크}{: FastAPI, Django, Laravel, React, Spring Boot} \\
            \textbf{개발 도구}{: Git, Docker, Kubernetes, Terraform, Redis, AWS, Bash, GraphQL, Jenkins, CI/CD, Prometheus, Grafana} \\
            \textbf{라이브러리}{: Django REST Framework, SQLModel, Pydantic, Uvicorn, Pytest, LangChain}
        }}
    \end{itemize}


%-------------------------------------------
\end{document}
